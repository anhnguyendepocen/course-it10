\documentclass[handout,intlimits]{beamer}
%\documentclass[intlimits]{beamer}
\usepackage[T1]{fontenc}
\usepackage{lmodern}
\usepackage[utf8]{inputenc}
\usepackage[ngerman,english]{babel}
\usepackage{fixltx2e}
\usepackage{graphicx}
\usepackage{eurosym}
\usepackage{booktabs}
\usepackage{dcolumn}
\usepackage{threeparttable}
\usepackage{threeparttablex}
\usepackage[sc]{mathpazo}
\usepackage{amsmath}
\usepackage{listings}
\usepackage[bitstream-charter]{mathdesign}
\usepackage{xcolor}
\usepackage{longtable}
\usepackage{dcolumn}
\usepackage[comma]{natbib}
\usepackage{verbatim}
\usepackage{hyperref}
\usepackage{ellipsis}
\usepackage{soul}
\usepackage{tabularx}
\usepackage{microtype}


% DEFINITIONS
\definecolor{izablue}{RGB}{54,91,160}
\definecolor{zewblue}{RGB}{42,107,183}
\definecolor{med-gray}{gray}{0.35}
\definecolor{izagray}{RGB}{220,220,220}
\definecolor{gray}{RGB}{242,242,242}
\newcommand{\beginbackup}{
	\newcounter{framenumberpreappendix}
	\setcounter{framenumberpreappendix}{\value{framenumber}}
}
\newcommand{\backupend}{
	\addtocounter{framenumberpreappendix}{-\value{framenumber}}
	\addtocounter{framenumber}{\value{framenumberpreappendix}}
}
\def\sym#1{\ifmmode^{#1}\else\(^{#1}\)\fi}
\def\lit#1{\textcolor{med-gray}{\tiny (#1)}}
\def\litc#1{\textcolor{med-gray}{\tiny\citep{#1}}}
\def\litf#1{\textcolor{med-gray}{\tiny\citep*{#1}}}
\setbeamercolor{structure}{fg=zewblue}
\def\blue#1{\textcolor{zewblue}{#1}}
\def\vect#1{\mathbold{#1}}
\lstset{basicstyle=\ttfamily}
\beamertemplatenavigationsymbolsempty
\setbeamerfont{itemize/enumerate subbody}{size=\footnotesize}
% / DEFINITIONS


% ZEW STUFF
\setbeamercolor{section in head/foot}{fg=white, bg=zewblue}
\setbeamertemplate{footline}{%
	\begin{beamercolorbox}[ht=3ex]{section in head/foot}%
		\hspace{1em}\insertshortauthor\,(\insertshortinstitute):\,\insertshorttitle\hfill%
		\insertframenumber\,/\,\inserttotalframenumber\kern1em\vskip0.5pt%
	\end{beamercolorbox}
}
\setbeamertemplate{headline}{%
	\begin{beamercolorbox}[ht=3ex]{section in head/foot}%
		\hspace{1em}\insertsection\hfill\kern0.5em\textcolor{zewblue}{g}\vskip0.5pt%
	\end{beamercolorbox}
}
\setbeamersize{text margin left=0.5em}
\setbeamersize{text margin right=1.5em}
% / ZEW STUFF



% BEGIN DOCUMENT
\begin{document}
\shorthandoff{"}


% DOCUMENT INFORMATION
\title{Redistribution in Microsimulation\\ Models with Behavioral Responses}
\author[Löffler]{Max Löffler}
\institute[ZEW]{ZEW and University of Cologne}
\date[January 14, 2015]{Canazei --- January 14, 2015}


% TITLE PAGE
\begin{frame}[plain]
	\titlepage
\end{frame}



%%%%%%%%%%%%%%%%%%%%%%%%%%%%%%%%%%%%%%%%%%%%%%%%%%
% BEGIN
%%%%%%%%%%%%%%%%%%%%%%%%%%%%%%%%%%%%%%%%%%%%%%%%%%


\section{Introduction}

\begin{frame}
\frametitle{What is this session about?}
\begin{itemize}
	\item \blue{Redistribution:} ``If top tax rate was 53 instead of 42\,\%, how would that affect net incomes, income inequality or work incentives?''
	\bigskip
	\item \blue{Microsimulation:} Ex ante analysis to evaluate counterfactual situations or policy reforms that are to be implemented in the future
	\bigskip
	\item \blue{Models:} Analysis starting from economic theory and theoretic models
	\bigskip
	\item \blue{Behavioral Responses:} Allows to study adjustments to changed incentives and predict responses to counterfactual situations
\end{itemize}
\end{frame}

\begin{frame}
\frametitle{Motivation}
\begin{itemize}
	\item \blue{Why could this be interesting for inequality and welfare analysis?}
	\begin{itemize}
		\item Understand and predict behavioral responses to redistributive policies
		\smallskip
		\item Estimation of structural utility models allows various welfare analyses \litf{bargain_etal_2013, bargain_etal_itax_2014, bargain_etal_annes_2014}
	\end{itemize}
	\bigskip
	\item \blue{Applied session}
	\begin{itemize}
		\item How do these models work?
		\smallskip
		\item How to estimate them?
		\smallskip
		\item How to predict behavior?
	\end{itemize}
	\bigskip
	\item \blue{Several references included} \litc{creedy_kalb_2005, creedy_kalb_2006, aaberge_colombino_2014}
\end{itemize}
\end{frame}

\begin{frame}
\frametitle{Outline}
\begin{enumerate}
	\item Short introduction and motivation
	\bigskip
	\item How to use these models?
	\begin{enumerate}
		\item Basic workflow using such models
		\smallskip
		\item Underlying theory and modeling assumptions
		\smallskip
		\item Data and choice set construction
		\smallskip
		\item How to estimate these models
		\smallskip
		\item Using these models for policy analyses
	\end{enumerate}
	\bigskip
	\item Model extensions
	\bigskip
	\item Applied session
	\bigskip
	\item References
\end{enumerate}
\end{frame}

\begin{frame}
\frametitle{Basic workflow}
\begin{enumerate}
	\item Specify a theoretical model of individual labor supply behavior
	\begin{itemize}
		\item E.g.\ $U=U\left(C,L\right)$, individuals maximize their utility over potential jobs
		\smallskip
		\item Jobs defined by wages $w$, working hours $h$ and other characteristics $\epsilon$
	\end{itemize}
	\bigskip
	\item Find suitable data and set up the choice set
	\begin{itemize}
		\item Identify chosen jobs (hours, wages and characteristics observed in the data)
		\smallskip
		\item Add counterfactual choices and simulate hypothetical hours and net incomes
	\end{itemize}
	\bigskip
	\item Estimate the model using this choice set, obtain preference estimates
	\bigskip
	\item Hold preferences constant and simulate counterfactual scenarios
	\begin{itemize}
		\item Study behavioral adjustments if tax/benefit system or wages were different
		\smallskip
		\item Calculate welfare measures (e.g.\ compensating variation, utility differentials)
	\end{itemize}
\end{enumerate}
\end{frame}


\section{Model}

\begin{frame}
\frametitle{Theoretical model}
\begin{itemize}
	\item \blue{Agents receive utility from consumption $C$ and leisure $L$}
	\begin{itemize}
		\item Mainly two different classes of models have been used over the last decades
		\smallskip
		\item Classic models assume maximization over continuous range of working hours, i.e.\ \blue{$\max_{h=[0,60]} U(C[h],L[h])$} \litc{burtless_hausman_1978, hausman_1981}
		\smallskip
		\item More recent models usually assume discrete choice between different jobs, i.e.\ \blue{$\max_{j\in J} U(C_j,L_j)$} \litf{aaberge_etal_1995, vansoest_1995, hoynes_1996}
		\smallskip
		\item Individual subscripts omitted, but no time subscripts: ``steady-state'' models
	\end{itemize}
	\bigskip
	\item \blue{Agents choose jobs and maximize utility.} How does it look like?
	\begin{itemize}
		\item Individual $n$'s utility of choosing job $j$ is defined as \blue{$U_{nj} = U\left(C_{nj},L_j,\epsilon_{nj}\right)$}
		\smallskip
		\item Researcher needs to identify form of $U$, distribution of $\epsilon_{nj}$ and job offers
		\smallskip
		\item Observed choices inform us about maximum utility and optimal working hours
	\end{itemize}
\end{itemize}
\end{frame}

\begin{frame}
\frametitle{Identification}
\begin{itemize}
	\item \blue{Wait, agents' labor supply is thus determined by consumption and leisure, both functions of working hours?} Isn't that tautological?
	\begin{itemize}
		\item Right. That's why econometricians care so much about identification
		\smallskip
		\item Non-parametric identification impossible with cross-sectional data \litc{manski_2014}
		\smallskip
		\item But even cross-sectional data typically gives us variation in wages, non-labor incomes, taxes and transfers (=consumption) for given working hours
	\end{itemize}
	\bigskip
	\item \blue{Use structural model and impose specific functional form}, e.g.
	\begin{itemize}
		\item $U_{nj} = \alpha_1 \ln C_{nj} + \alpha_2 (\ln C_{nj})^2 + \alpha_3 \ln L_j + \alpha_4 (\ln L_j)^2 + \alpha_5  \ln C_{nj} \ln L_j + \epsilon_{nj}$
		\smallskip
		\item $C_{nj}$ accounts for earnings $w_{nj}L_j$, non-labor income $I$ and tax/benefit system
		\smallskip
		\item Assumes homogeneous preferences and specific functional form
	\end{itemize}
\end{itemize}
\end{frame}

\begin{frame}
\frametitle{Job offers and hours categories}
\begin{itemize}
	\item \blue{We assume that individuals choose among jobs, but what is a job?}
	\begin{itemize}
		\item A job $j$ is defined by hours $h_j$, wages $w_{nj}$ and unobservables $\epsilon_{nj}$
		\smallskip
		\item Most often, job modeling boils down to three simple assumptions:
		\begin{enumerate}
			\item Unobservables follow Gumbel distribution: \blue{$\epsilon_{nj}\sim GEV(0,1,0)$}
			\smallskip
			\item Wages assumed exogenous to hours or preferences: \blue{$w_{nj}=w_n$}
			\smallskip
			\item Hours discretized to interval means: \blue{$h_j\in[0, 10, 20, 30, 40, 50, 60]$}
		\end{enumerate}
		\litf{vansoest_1995, hoynes_1996, keane_moffitt_1998, euwals_vansoest_1999, blundell_etal_2000, vansoest_etal_2002, bargain_etal_2014}
		\smallskip
		\item This simple model generates seven choices for each individual
	\end{itemize}
	\bigskip
	\item \blue{Modeling approach is rather restrictive.} Alternative approaches:
	\begin{itemize}
		\item Define random choice-set instead of fixed hours \litf{aaberge_etal_2009}
		\smallskip
		\item Allow correlation between wages and preferences \litf{blundell_shephard_2012}
		\smallskip
		\item Allow hours-dependent wages and flexible correlation \litf{loeffler_etal_2014}
	\end{itemize}
\end{itemize}
\end{frame}


\section{Data}

\begin{frame}
\frametitle{Data sources}
\begin{itemize}
	\item \blue{What kind of data is required for the estimation?}
	\begin{itemize}
		\item The model requires information at least on wages and working hours
		\smallskip
		\item Ideally, the data also contains household characteristics, information on children, demographics, social background, occupational information, \dots
		\smallskip
		\item Most models based on cross-sectional data, but panel data highly useful
	\end{itemize}
	\bigskip
	\item \blue{Which data sets are used in practice?}
	\begin{itemize}
		\item Usually household surveys, only rarely administrative data because working hours are seldom reported in tax registers or social security records
		\smallskip
		\item EU: SILC, US: CPS, PSID, NLSY, Germany: SOEP, Britain: BHPS, \dots
	\end{itemize}
\end{itemize}
\end{frame}

\begin{frame}
\frametitle{Tax/benefit calculator}
\begin{itemize}
	\item \blue{What we need to do with the data}
	\begin{itemize}
		\item Clean up sample and keep only working-age population (or subgroups)
		\smallskip
		\item Create choice set according to job model (for now: simple approach)
		\smallskip
		\item Calculate individual wage rates (and impute wage for non-workers)
		\smallskip
		\item Duplicate individual observations by number of hypothetical choices
	\end{itemize}
	\bigskip
	\item \blue{How to create hypothetical job choices?}
	\begin{itemize}
		\item Define weekly/monthly working hours for every hypothetical job choice
		\smallskip
		\item Based on hypothetical hours and wages, calculate gross earnings for all jobs
		\smallskip
		\item Apply tax/benefit calculator to these hypothetical households with their hypothetical working hours and gross earnings to get disposable income
		\smallskip
		\item Popular calculators: EUROMOD for Europe, TAXSIM for the US, \dots
	\end{itemize}
	\bigskip
	\item \blue{Observed working hours inform us about the job actually chosen}
\end{itemize}
\end{frame}

\begin{frame}
\frametitle{Browsing through the data}
\begin{itemize}
	\item The resulting data set should look somehow like this one:
\end{itemize}
\begin{table}[htbp]
	\centering\tiny
	\begin{threeparttable}
	\begin{tabular}{rrrrrrrrr}
	\toprule
	\textbf{\#} & \textbf{id} & \textbf{hours} & \textbf{wage} & \textbf{gross} & \textbf{dpi} & \textbf{leisure} & \textbf{choice} & \textbf{\dots} \\
	\midrule
	1	&	1	&	0	&	7.5	&	0	&	400	&	80	&	0	& \dots \\
	2	&	1	&	10	&	7.5	&	300	&	500	&	70	&	0	& \dots \\
	3	&	1	&	20	&	7.5	&	600	&	600	&	60	&	1	& \dots \\
	4	&	1	&	30	&	7.5	&	900	&	700	&	50	&	0	& \dots \\
	5	&	1	&	40	&	7.5	& 1,200	&	800	&	40	&	0	& \dots \\
	6	&	1	&	50	&	7.5	& 1,500	&	900	&	30	&	0	& \dots \\
	7	&	1	&	60	&	7.5	& 1,800	& 1,000	&	20	&	0	& \dots \\
	\midrule
	8	&	2	&	0	&	9.7	&	0	&	400	&	80	&	0	& \dots \\
	9	&	2	&	10	&	9.7	& 388	&	550	&	70	&	0	& \dots \\
	10	&	2	&	20	&	9.7	& 766	&	700	&	60	&	0	& \dots \\
	11	&	2	&	30	&	9.7	& 1,164	&	850	&	50	&	0	& \dots \\
	12	&	2	&	40	&	9.7	& 1,552	& 1,000	&	40	&	1	& \dots \\
	13	&	2	&	50	&	9.7	& 1,940	& 1,150	&	30	&	0	& \dots \\
	14	&	2	&	60	&	9.7	& 2,328	& 1,300	&	20	&	0	& \dots \\
	\midrule
	\dots	&	\dots	&	\dots	&	\dots	& \dots	& \dots	&	\dots	&	\dots& \dots \\
	\bottomrule
	\end{tabular}
	\end{threeparttable}
\end{table}
\end{frame}


\section{Estimation}

\begin{frame}
\frametitle{Estimation approach}
\begin{itemize}
	\item \blue{Bringing the theory to the data}
	\begin{itemize}
		\item We know the choice and we know the consumption and leisure for all jobs
		\smallskip
		\item We want to outback the preference coefficients $\vect{\alpha}$ of our utility function
	\end{itemize}
	\bigskip
	\item \blue{Recall the utility function we assumed for our model}
	\begin{itemize}
		\item $U_{nj} = \underbrace{\alpha_1 \ln C_{nj} + \alpha_2 (\ln C_{nj})^2 + \alpha_3 \ln L_j + \alpha_4 (\ln L_j)^2 + \alpha_5  \ln C_{nj} \ln L_j}_{V_{nj}} + \epsilon_{nj}$
		\smallskip
		\item Moreover we assumed that $\epsilon_{nj}$ is extreme value distributed
		\smallskip
		\item Yields well known multinomial or conditional logit problem \litc{mcfadden_clogit_1974}
		\smallskip
		\item $P\left(U_{nj}>U_{nk}\forall k\neq j\right) = \frac{\exp V_{nj}}{\sum_{h\in J}\exp V_{nh}} \quad \forall \quad n=1,\dots,N, \quad j\in J$
		\smallskip
		\item We can estimate this model using maximum likelihood methods
	\end{itemize}
\end{itemize}
\end{frame}

\begin{frame}[fragile]
\frametitle{Estimation using Stata}
\begin{itemize}
	\item \blue{Define the likelihood function of this simple model}
	\begin{itemize}
		\item Log-likelihood given by: $\ln L=\sum_{n=1}^N\sum_{j\in J}y_{nj}\left(V_{nj} - \ln\sum_{h\in J}\exp V_{nh}\right)$
		\smallskip
		\item Where $y_{nj}=1$ if individual $n$ chose alternative $j$ and $y_{nj}=0$ otherwise
	\end{itemize}
	\bigskip
	\item \blue{Turning to Stata}
	\begin{itemize}
		\item The model defined above can be estimated using Stata's \verb+clogit+ command
		\smallskip
		\item Call by: \verb+clogit choice ln_dpi ln_leisure ... [fw=...], group(id)+
	\end{itemize}
	\bigskip
	\item \blue{Rather restrictive when it comes to more complicated models}
	\begin{itemize}
		\item User-written command \verb+lslogit+ for labor supply estimation \litc{loeffler_2013}
		\smallskip
		\item Supports often used specifications and extensions \litf{loeffler_etal_2014}
		\smallskip
		\item Call by: \verb+lslogit choice [fw=...], group(id) con(dpi) lei(leisure)+
	\end{itemize}
\end{itemize}
\end{frame}

\begin{frame}[fragile]
\frametitle{Example output}
\begin{center}
	\tiny
	\begin{verbatim}
        . lslogit choice, group(id) c(dpi) l(leisure) boxcox cx(age*_m) lx1(age*_m)

        Mixed Logit Labor Supply Model                    Number of obs   =       5761
                                                          LR chi2(2)      =     171.99
                                                          Prob > chi2     =     0.0000
        Log likelihood = -1368.2779                       Pseudo R2       =     0.1456

                                             (Std. Err. adjusted for clustering on id)
        ------------------------------------------------------------------------------
              choice |      Coef.   Std. Err.      z    P>|z|     [95% Conf. Interval]
        -------------+----------------------------------------------------------------
        Cx     age_m |   63.48062   10.78075     5.89   0.000     42.35074     84.6105
              age2_m |  -8.688809   1.483041    -5.86   0.000    -11.59552   -5.782102
               _cons |  -112.7283   19.40653    -5.81   0.000    -150.7644   -74.69218
        -------------+----------------------------------------------------------------
        CxL1   _cons |   .0862592   .0578086     1.49   0.136    -.0270435    .1995619
        -------------+----------------------------------------------------------------
        L1x    age_m |   1.314325   1.334258     0.99   0.325    -1.300773    3.929424
              age2_m |  -.1787296   .1835297    -0.97   0.330    -.5384413    .1809821
               _cons |   -2.32873    2.38518    -0.98   0.329    -7.003596    2.346136
        -------------+----------------------------------------------------------------
                /l_C |    .593499   .0875811     6.78   0.000     .4218433    .7651548
               /l_L1 |  -2.624566   .5228987    -5.02   0.000    -3.649429   -1.599704
        -------------+----------------------------------------------------------------
             [dudes] |   .0341955
        ------------------------------------------------------------------------------
        Model: - Box-Cox utility function
        
        .
	\end{verbatim}
\end{center}
\end{frame}


\section{Prediction}

\begin{frame}[fragile]
\frametitle{Postestimation}
\begin{itemize}
	\item \blue{Maximum likelihood gives us estimates for preference coefficients $\vect{\hat{\alpha}}$}
	\begin{itemize}
		\item This allows us to predict utility levels $\hat{V}_{nj}$ from choosing the different jobs
		\begin{itemize}
			\item Call by: \verb+predict util, xb+
		\end{itemize}
		\smallskip
		\item Using these we can predict choice probabilities or utility maximizing choice
		\begin{itemize}
			\item Call by: \verb+predict prob, pc1+
		\end{itemize}
		\smallskip
		\item Enables us to check estimation fit by comparing observed and predicted choices
	\end{itemize}
\end{itemize}
\begin{table}[htbp]
	\centering\tiny
	\begin{threeparttable}
	\begin{tabular}{rrrrrrr}
	\toprule
	\textbf{\#} & \textbf{id} & \textbf{hours} & \textbf{choice} & \textbf{util} & \textbf{prob} & \textbf{\dots} \\
	\midrule
	1	&	1	&	0	&	0	& 	1.0	&	0.100	& \dots \\
	2	&	1	&	10	&	0	& 	1.5	&	0.165	& \dots \\
	3	&	1	&	20	&	1	& 	2.0	&	0.271	& \dots \\
	4	&	1	&	30	&	0	& 	1.5	&	0.165	& \dots \\
	5	&	1	&	40	&	0	& 	1.0	&	0.100	& \dots \\
	6	&	1	&	50	&	0	& 	1.0	&	0.100	& \dots \\
	7	&	1	&	60	&	0	& 	1.0	&	0.100	& \dots \\
	\midrule
	\dots	&	\dots	&	\dots	&	\dots	& \dots	&	\dots 	& \dots \\
	\bottomrule
	\end{tabular}
	\end{threeparttable}
\end{table}
\end{frame}

\begin{frame}[fragile]
\frametitle{Counterfactual analysis}
\begin{itemize}
	\item \blue{Holding preferences $\vect{\hat{\alpha}}$ constant, we can simulate policy reforms}
	\begin{itemize}
		\item E.g.\ by calculating new disposable incomes for all job categories
		\begin{itemize}
			\item \verb+gen dpi2 = ...+
			\item \verb+replace dpi = dpi2+
			\item \verb+predict prob2, pc1+
		\end{itemize}
	\end{itemize}
\end{itemize}
\begin{table}[htbp]
	\centering\tiny
	\begin{threeparttable}
	\begin{tabular}{rrrrrrrrrr}
	\toprule
	\textbf{\#} & \textbf{id} & \textbf{hours} & \textbf{dpi} & \textbf{dpi2} & \textbf{choice} & \textbf{prob} & \textbf{prob2} & \textbf{\dots} \\
	\midrule
	1	&	1	&	0	&	400	&	300	&	0	& 0.100	&	0.07	& \dots \\
	2	&	1	&	10	&	500	&	400	&	0	& 0.165	&	0.10	& \dots \\
	3	&	1	&	20	&	600	&	500	&	1	& 0.271	&	0.14	& \dots \\
	4	&	1	&	30	&	700	&	700	&	0	& 0.165	&	0.14	& \dots \\
	5	&	1	&	40	&	800	&	900	&	0	& 0.100	&	0.20	& \dots \\
	6	&	1	&	50	&	900	& 1,100	&	0	& 0.100	&	0.18	& \dots \\
	7	&	1	&	60	& 1,000	& 1,300	&	0	& 0.100	&	0.18	& \dots \\
	\midrule
	\dots	&	\dots	&	\dots	&	\dots	& \dots	& \dots	&	\dots	&	\dots& \dots \\
	\bottomrule
	\end{tabular}
	\end{threeparttable}
\end{table}
\end{frame}


\section{Conclusion}

\begin{frame}
\frametitle{Model extensions}
\begin{itemize}
	\item \blue{These are the basics. Model can be extended in several dimensions}
	\begin{itemize}
		\item Alternative representation of job offers \litc{aaberge_etal_2009, dagsvik_etal_2014}
		\smallskip
		\item Different assumptions on the wage distribution and exogeneity \litc{loeffler_etal_2014}
		\smallskip
		\item Functional form of the utility function \litc{vansoest_etal_2002, loeffler_etal_2014}
		\smallskip
		\item Part-time restrictions and fixed costs \litc{vansoest_1995, euwals_vansoest_1999}
		\smallskip
		\item Random preferences and unobserved heterogeneity \litc{vansoest_1995, pacifico_2013}
		\smallskip
		\item Correlated wages and preferences \litc{breunig_etal_2008, blundell_shephard_2012}
		\smallskip
		\item Welfare stigma from benefit participation \litc{hoynes_1996, keane_moffitt_1998}
		\smallskip
		\item Job offers and different economic sectors \litc{dagsvik_strom_2006}
	\end{itemize}
	\bigskip
	\item \blue{Huge literature, but all follow the same outline and workflow}
	\begin{itemize}
		\item Good intro: ``Labour supply and microsimulation'' \litc{creedy_kalb_2005, creedy_kalb_2006}
		\smallskip
		\item Handbook chapter: ``Labour Supply Models'' \litc{aaberge_colombino_2014}
		\smallskip
		\item Comprehensive sensitivity check with respect to assumptions \litc{loeffler_etal_2014}
	\end{itemize}
\end{itemize}
\end{frame}


\beginbackup

\begin{frame}
\bigskip\bigskip
\href{https://github.com/mloeffler/course-it10/archive/master.zip}{\blue{\,\nolinkurl{https://github.com/mloeffler/course-it10/archive/master.zip}}}

\begin{center}\bigskip\bigskip
	Comments or questions? --- \href{mailto:loeffler@zew.de}{\blue{\nolinkurl{loeffler@zew.de}}}
\end{center}
\end{frame}


% APPENDIX
%\appendix
%
%\begin{frame}
%\begin{center}
%	{\LARGE Appendix}
%\end{center}
%\end{frame}
%
%
%\section{Appendix}
%
%\begin{frame}[label=literature]
%\frametitle{Existing literature
%		\hyperlink{motivation<1>}{\beamerreturnbutton{motivation}}}
%\begin{itemize}
%	\item Long and extensive literature on the \blue{capitalization} of property taxes into house values and the \blue{tax shifting} on housing rents
%	\begin{itemize}
%		\item Different views on capitalization and incidence: ``capital tax view'' vs.\ ``benefit tax view'' \litc{marshall_1890,edgeworth_1897,bickerdike_1902,simon_1943,tiebout_1956,mieszkowski_1972,hamilton_1976,mieszkowski_zodrow_1989,fischel_1992,zodrow_2001,zodrow_2001_lincoln}
%		\hyperlink{views<1>}{\beamergotobutton{property tax views}}
%		\smallskip
%		\item Some empirical evidence for tax shifting on housing rents \litc{orr_1968,orr_1970,orr_1972,heinberg_oates_1970,hyman_pasour_1973,dusansky_etal_1981,carroll_yinger_1994}
%	\end{itemize}
%	\bigskip
%	\item \blue{No theoretical or empirical consensus} on the property tax incidence
%	\bigskip
%	\item Empirical studies on property tax incidence mainly for the US
%	\begin{itemize}
%		\item Usually one year cross-sectional data, less than 100 observations
%		\smallskip
%		\item Estimates range between 0-115\,\%, \blue{identification rather complicated}
%	\end{itemize}
%	\bigskip
%	\item One study for Baden-Württemberg, finds no shifting on rents \litc{buettner_2003}
%\end{itemize}
%\end{frame}
%
%\begin{frame}[label=views]
%\frametitle{Property tax views
%		\hyperlink{motivation<1>}{\beamerreturnbutton{motivation}}}
%\begin{itemize}
%	\item \blue{Traditional view} \litc{edgeworth_1897,simon_1943,netzer_1966}
%	\begin{itemize}
%		\item Tax introduced in single municipality, perfectly elastic capital supply
%		\smallskip
%		\item Tenants bear the full tax burden of property taxation
%	\end{itemize}
%	\bigskip
%	\item \blue{Capital tax view} \litc{mieszkowski_1972,mieszkowski_zodrow_1989}
%	\begin{itemize}
%		\item Extends ``old view'' with Harberger general equilibrium model
%		\smallskip
%		\item Capital owners bear the national average burden of property taxes
%	\end{itemize}
%	\bigskip
%	\item \blue{Benefit tax view} \litc{tiebout_1956,oates_1969,hamilton_1976,fischel_1992}
%	\begin{itemize}
%		\item Households choose ``optimal municipality'', competing communities
%		\smallskip
%		\item Property taxes to finance local public goods, non-distortional
%	\end{itemize}
%	\bigskip
%	\item<2-> \blue{Hard to provide exclusive evidence for different views} \litc{fischel_etal_2011}
%	\begin{itemize}
%		\item General equilibrium aspects hard to pin down empirically
%		\smallskip
%		\item Empirics focused on partial analysis (as in corporate tax literature)
%	\end{itemize}
%\end{itemize}
%\end{frame}
%
%\begin{frame}[label=taxrates]
%\frametitle{Federal tax rates (in \%)
%		\hyperlink{institutions<1>}{\beamerreturnbutton{institutions}}}
%\begin{table}[htbp]
%	\centering\tiny
%	\begin{threeparttable}
%	\begin{tabular}{@{}l@{}rrlrr@{}r@{}}
%	\toprule
%	\multicolumn{2}{c}{\textbf{West Germany}} & & \multicolumn{4}{c}{\textbf{East Germany}}\\
%	\cmidrule(r){1-2}\cmidrule(l){4-7}
%						&		&&	& \multicolumn{3}{c}{Tax rate by population 1933} \\
%	 \cmidrule(l){5-7}
%	Building type 			& Tax rate && Building type		& <25k	& 25k-1,000k	& >1,000k \\
%	\midrule
%				 		& 		&& \multicolumn{4}{c}{Built before 1924} \\
%	 \cmidrule(l){4-7}
%	 One-family houses		& 		&& One-family houses		& 	&	&	 \\
%	\qquad First 38,347 EUR	& 0.26	&& \qquad First 15,339 EUR	& 1.0	& 0.8	& 0.6 \\
%	\qquad Additional value	& 0.35	&& \qquad Additional value		& 1.0	& 1.0	& 1.0 \\
%	Two-family houses		& 0.31 	&& Other houses			& 1.0	& 1.0	& 1.0 \\
%				 		& 		&& \multicolumn{4}{c}{Built after 1924} \\
%	 \cmidrule(l){4-7}
%						& 	 	&& One-family houses		&	& 	& 	 \\
%						& 	 	&& \qquad First 15,339 EUR	& 0.8	& 0.6	& 0.5 \\
%						& 	 	&& \qquad Additional value		& 0.8	& 0.7	& 0.6 \\
%						& 	 	&& Other houses			& 0.8	& 0.7	& 0.6	 \\
%				 		& 		&& \multicolumn{4}{c}{Vacant lots} \\
%	 \cmidrule(l){4-7}
%						& 	 	&& Business purpose			& 1.0	& 1.0	& 1.0 \\
%	Other houses/vacant lots	& 0.35	&& Other					& 0.5	& 0.5	& 0.5 \\
%	\bottomrule
%	\end{tabular}
%	\begin{tablenotes}
%	\item \emph{Source:} §§ 15, 41 \emph{Grundsteuergesetz}, §§ 29-33 \emph{Grundsteuerdurchführungsverordnung}.
%	\end{tablenotes}
%	\end{threeparttable}
%\end{table}
%\end{frame}
%
%\begin{frame}[label=equilibrium]
%\frametitle{Spatial equilibrium
%		\hyperlink{predictions<1>}{\beamerreturnbutton{predictions}}}
%\begin{itemize}
%	\item These supply and demand equations define the equilibrium:
%	\begin{itemize}
%		\item \blue{Housing demand and labor supply}: $\ln N_c^{LS} = \ln N_c^{HD} = \frac{\ln w_c}{s} - \gamma \frac{\ln r_c}{s} - \gamma \frac{\ln (1+t_c)}{s} + \frac{\ln A_c}{s} - \ln {C} - \ln{\pi}$
% 		\smallskip
%		\item \blue{Inverse labor demand}: $\ln w_c = a_0 + \tilde{X}_c + \eta \ln N_c^{LD}$
%		\smallskip 
%		\item \blue{Inverse housing supply}: $\ln r_c =\ln Z_c + k_c \ln N_c^{HS}$
%	\end{itemize}
%	\bigskip
%	\item \blue{Spatial equilibrium} can be described by the following quantities:
%	\begin{itemize}
%		\item $\ln N_c^{\ast} = \frac{a_0 - s\bar{U}  - \gamma \ln (1+t_c)  + \tilde{X}_c + \ln A_c - \gamma \ln Z_c}{s + \gamma k_c - \eta}$
%		\smallskip
% 		\item $\ln w_c^{\ast} = \frac{(s + \gamma k_c)(a_0 + \tilde{X}_c) + \eta (\ln A_c - \gamma \ln (1+t_c) -\gamma \ln Z_c-s\bar{U})}{s + \gamma k_c - \eta}$
%		\smallskip
% 		\item $\ln r_c^{\ast} = \frac{(s - \eta) \ln Z_c + k_c (a_0 - s \bar{U} - \gamma \ln (1+t_c) + \tilde{X}_c + \ln A_c)}{s + \gamma  k_c - \eta}$
%	\end{itemize}
%\end{itemize}
%\end{frame}


\section{References}

\begin{frame}[t,allowframebreaks]
\frametitle{References}
\scriptsize
\bibliography{bibrefs}
\bibliographystyle{dcu}
\end{frame}

\backupend

\end{document}
